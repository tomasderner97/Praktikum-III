\documentclass{protokol}
\leftheader{Studium ohybových jevů v laserovém svazku}
\centerheader{}
\rightheader{Tomáš Derner}

\begin{document}

  \section*{Úkol}

    \begin{enumerate}
      \item Ze změřeného ohybového obrazce zobrazeného na milimetrovém papíru určete mřížkovou konstantu mřížky.
      \item Pomocí aparatury proměřte ohybové obrazce: mřížky, štěrbiny a dvojštěrbiny. Konkrétní difrakční prvky vybere vyučující. Zpracováním měření určete parametry použitých difrakčních prvků.
      \item Okalibrujte mikroskopový okulár s použitím metody lineární regrese, odhadněte relativní chybu kalibrace.
      \item Mikroskopem změřte parametry všech použitých difrakčních prvků.
      \item Výsledky měření v úkolech č.1, č.2 a č.4 srovnejte a diskutujte, v kterém případě jsou spočtené parametry zatíženy nejmenší chybou. 
    \end{enumerate}

  \section*{Teorie}

    V tomto praktiku měříme ohyb laserového svazku způsobený difrakční mřížkou a štěrbinami. Protože použitý laser má poměrně velkou divergenci svazku, použijeme v měření spojnou čočku, viz \cite{mereni}. 

    Pro získání mřížkové konstanty $a$ využijeme vztahu pro úhel $\varphi$ mezi dvěma body maximální intenzity difrakčního obrazce
    \begin{equation}
      \varphi = \frac{\lambda}{a},
    \end{equation} 
    kde $\lambda$ je vlnová délka použitého světla. Úhel $\varphi$ získáme z rovnice
    \begin{equation}
      \varphi = \frac{x}{l},
    \end{equation}
    kde $x$ je vzdálenost dvou maxim a $l$ vzdálenost difrakčního obrazce od spojné čočky.

  \section*{Výsledky}

    \subsection*{Úkol 1}

    \subsection*{Úkol 2}

    \subsection*{Úkol 3}

    \subsection*{Úkol 4}

  \section*{Diskuse}

  \section*{Závěr}

  \begin{thebibliography}{}

    \bibitem{mereni}
    Pokyny k měření "", dostupné z\\ \url{}, .\,.\,
  
  \end{thebibliography}

\end{document}