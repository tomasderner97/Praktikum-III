\documentclass{protokol}
\leftheader{Charakteristiky optoel. součástek}
\centerheader{Praktikum III}
\rightheader{Tomáš Derner}

\begin{document}

  \section*{Úkol}

    \begin{enumerate}
      \item Na internetu najděte katalogové listy všech optoelektronických součástek, které budete v úloze používat, konkrétní měřené součástky vybere vyučující. Parametry důležité ke splnění pracovních úkolů vypište a přiložte do zápisu z měření.
      \item Změřte voltampérové a světelné charakteristiky dvou luminiscenčních diod v propustném směru. Grafy vytvořte v praktiku, jsou povinnou součástí zápisu z měření.
      \item Ze změřených V-A charakteristik určete pro jednotlivé diody statický odpor $R_d$, dynamický odpor $R_{di}$, hodnotu konstanty n a prahové napětí $U^*$. Určete, z jakého materiálu jsou jednotlivé diody zhotoveny. Nezapomeňte na graf $\ln(I_F)$ vs. $U_F$.
      \item Změřte charakteristiky fototranzistoru při třech různých hladinách osvětlení. Měření proveďte pomocí pikoampérmetru s vestavěným zdrojem Keithley s funkcí ukládání dat do paměti přístroje. Povinnou součástí zápisu z měření jsou grafy naměřených charakteristik, tabulky do protokolu netiskněte.
      \item Změřte zisk fototranzistoru.
    \end{enumerate}

  \section*{Teorie}

    

  \section*{Výsledky}

  \section*{Diskuse}

  \section*{Závěr}

  \begin{thebibliography}{}

    \bibitem{mereni}
    Pokyny k měření "", dostupné z\\ \url{}, .\,.\,
  
  \end{thebibliography}

\end{document}