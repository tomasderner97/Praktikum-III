\documentclass{protokol}
  \leftheader{Jednoduché aplikace interference}
  \centerheader{Praktikum III}
  \rightheader{Tomáš Derner}

  \begin{document}

    \section*{Sth}

    \begin{enumerate}
      \item Změřte tloušťku tenké vrstvy ve dvou různých místech. Vyhodnoťte získané tloušťky a diskutujte, zda je vrstva v rámci chyby nepřímého měření na obou místech stejně silná.
      \item Pomocí Newtonových interferenčních kroužků změřte oba poloměry křivosti u dvou vybraných čoček. Chybu v určení poloměru křivosti stanovte z vhodně použité lineární regrese.
      \item Kontrolu vámi provedené kalibrace stupnice okuláru proveďte metodou postupných měření a zpracujte lineární regresí.
      \item Výsledky měření v bodě 2 porovnejte s optickou mohutností čoček změřenou pomocí fokometru. Index lomu materiálu měřených čoček je $1,523$.
    \end{enumerate}

    Rovnice:~\eqref{eq:sth}

    \input{sth}
    
  \end{document}